%\mychapter{3}{methodology}

\section{Current Project}

\subsection{Our framework}

We propose a novel method of pre-processing spike times by looking at the time since the previous spike, which we refer to as the previous time.  Currently, our pre-processing technique is  applied to synthetic data as a first step towards being
able to pre-process  real-world data using the previous time.  To the best of our knowledge, using previous time  to pre-process spike train data from the CA1 region of the rat hippocampus has never been done before.\\

Our preliminary results on synthetic experiments  show  that  previous time data  capture  the  four and a half laps taken by a rat  around a simulated circular track.  We use the diffusion maps algorithm \cite{coifman2006diffusion}  to obtain a low dimensional model of the simulated data and compare the algorithm's performance with that of PCA.\\

Traditionally, spike trains are often smoothed using a suitable kernel (see section 3.4.2) in order to obtain an estimate
of the  firing rate.  Smoothing a simulated firing rate with a kernel such as a Gaussian or decaying exponential  and then adding additive and Poisson noise to the simulated firing rate could be an ideal way to generate synthetic data which closely mimics the nature of the real-world  data we have. However  there are many parameters to tune when using kernels.
Moreover,  we do not have  a suitable measure of error for  analyzing the performance  of the diffusion maps  algorithm and PCA on our synthetic data. To work around these two obstacles, we simulate clean firing rate data which perfectly encodes the position of the animal on a circular track and compare the output of the  two algorithms on both the clean firing rate data and the noisy previous time data. \\

Our motivation for using both  firing rate data  and previous time data  is based on the fact that the position of the rat is coded by place cells using both the firing rate and the precise time at which the cells fire with respect to the hippocampal theta rhythm. The theta rhythm is a sinusoid of frequency 7-12 Hz which occurs whenever a rat changes position in  a specific direction \cite{OKeefe1971, Burgess1993}.\\


In section  4.2 and 4.3,  we define the previous time function and illustrate it graphically.  In addition, we give a description of our simulation and analyzes used. In section 5, we show our results of diffusion maps and PCA on synthetic data sets.
Finally in section 6, we give a brief discussion on our results and suggest possible future directions.


\subsection{Nature of our raw real-world data}
The  raw real-world data set consists  of multiple single-unit single-trial spike trains,\\
\[ 
\text{T}^{\text{spike}} = \displaystyle \{ \{ t^{j} \} ,  1 \leq j \leq 32 \}  
\]
recorded from 32 neurons, called  place cells, in the CA1 region of the rat hippocampus, where,
$\displaystyle  \{t^{j}\} =  \{t_{1}^{j}, ....., t_{n_{i}}^{j} \} $, represents  a sequences of n$_{i}$ recorded times at which the spikes of the $j^{th}$ neuron occurred. \\
However, we only show results based on synthetic data due to lack of a suitable measure of error to quantify 
the performance of  diffusion maps and PCA on real world data.  We pre-process the simulated spike trains by computing  the previous time  defined in section 4.3.
 

%%%%%%% %%%  deleted this because our firing rate does not have spikes   %%%%%%%%%%%%%%%%%%%%%%%%%%
%\subsubsection{Conversation to a firing rate}
%For each neuron labeled $j$,  we smooth the corresponding spike train T$^{\text{spike}}_{j}(t)$ to obtain the firing rate function $R^{j}(t).$ The  firing rate function, for the $j^{th}$ neuron is given by
%\begin{equation} \label{jfirerate}
%\text{R}^{j}(t) = \displaystyle \sum_{i=1}^{n_{i}}  \frac{1}{\sigma \sqrt{2\pi}} 
%e^{-\dfrac{(t_{i_{k}}^{j}  - t_{i_{l}}^{j})^2}{2\sigma^2}}. 
%\end{equation}
%%%%%%%%%%%%%%%%%%%%%%%%%%%%%%%%%%%%%%%%%%%%%%%%%%%%%%%%%%%%%%%%%%%%%%%%

\subsection{The previous time function}

Given any time $t$, define the previous time function of the $j^{th}$ neuron, denoted by $\text{P}^{j}(t)$ as follows. Let 
\[
t^{j}_{\text{prev}}(t) = \displaystyle \max  \{  t^{j}_{i} \ \ | \ \ t^{j}_{i} < t, 1 \leq i \leq n_{i} \} \ \ \text{for all spike trains} \quad  \{t^{j}\} \in T^{\text{spike}}, 
\]
where $t^j_i$ denotes the $i^{th}$ spike of the $j^{th}$ neuron.  The previous time function, P$^{j}(t)$, is given by, 
\begin{equation}\label{prevtimefun}
\text{P}^{j}(t) = t^{j}_{\text{prev}}(t) - t.
\end{equation}
Previous time refers to the time since the last spike of a given neuron. The previous time function captures the history of the neuron's  activity even in the absence of spikes.

Figure \ref{fig:PrevTime} below illustrates how the previous time function is obtained from the firing activity of a given neuron.


 \begin{figure}[H]
        \centering
          \includegraphics[width=\textwidth]{./images/PrevTime_graph.pdf}
           \caption[]
            {\small The previous time function $\text{P}^{j}(t) = t^{j}_{\text{prev}}(t) - t,$ based on a raster plot of a single neuron.   $P^{j}(t)$ is equal to zero when a neuron fires a spike  and then moves away from zero with a slope of negative one to a height equal to the difference between two adjacent spike times, below the horizontal axis. } 
             \label{fig:PrevTime}
  \end{figure}


\subsection{Simulation}
We consider a population of $N=32$ sensory neurons (place cells) encoding a 
one-dimensional circular stimulus variable, $\theta(t)$, which represents 
the position of a rat, at any time $t$ along a circular track, during a behavioral task.
We let $\theta(t) = c t \ \ (\text{mod} \ \ 2\pi)$, where time $t$, is in seconds, $c = \dfrac{2\pi}{T_{\text{lap}}}$ is the speed of the animal, and
$\text{T}_{\text{lap}}$ is the time taken for the animal to make one lap around the circle.
From  $\theta(t)$,  we model the place field of the $j^{th}$ neuron using a Gaussian 

\begin{equation}
{g}^{j}(\theta) = \displaystyle  f_{\text{bg}} + f_{\max} 
\exp\bigg(-\dfrac{\text{dist}^{2}(\theta - \phi_{j})}{2\epsilon_{j}^{2}} \bigg)
\end{equation}

where $f_{bg}$ is the background firing rate that is independent of the underlying stimulus, $f_{\max}$ denotes the maximum firing rate of each neuron, $\epsilon_j \in  [0.01, 1]$ is the width of the $j^{th}$ neuron's place field  and $\phi_{j}$ is the preferred position of the $j^{th}$  neuron or the center of the j$^{th}$ place field. We assume that the centers of the place fields are distributed uniformly in space.  In particular, we set 
$$\phi_j = \frac{j2\pi}{N}, \ \    \text{for}   \ \    1 \leq j \leq N.$$
Intuitively, $g^{j}(\theta)$, models  the likelihood that the rat's neuron fires given the animal's position $\theta(t)$, in space.
The Gaussian function indicates that the rat's place cell is  likely to fire maximally  at the center of the cell's place field
which mimics a characteristic  of a real-world rat's place cells.
Our model assumes that each neuron in the simulated rat's brain has a single place field. This is a realistic assumption as 
most place cells have one place field (see section 6 on possible future directions). \\

Given any two angles, $\theta_{1}, \theta_{2} \in [0, 2\pi]$, the quantity
$\text{dist}(\theta_{1}, \theta_{2})$ denotes the shortest distance between 
two points on a unit circle, given by
\[
\text{dist}(\theta_{1}, \theta_{2}) = \abs{\big( \big( (\theta_{1} - \theta_{2}) + \pi  \big) \ \ \text{mod} \ \ 2\pi \big) - \pi}.
\]

Our simulation assumes  that information about the animal's environment is  encoded by the firing rate function
$$R^{j}(t) =  \text{g}^{j}(\theta(t)).$$
Thus given a rat position $\theta(t)$, all the 32  neurons  in the simulated rat's brain fire according to the firing rate function  $R^{j}(t).$


\subsection{Analysis}
We set the start  and end time of the synthetic experiment  to be  $0$ and $T$ respectively. We then divide the interval
$[0, T],$ into equal sub-intervals  of width $\Delta t$.
We carry out  our analyzes  using two different data sets:  a clean firing rate data set and  a noisy previous time data set.
We use two data sets because we do not have a suitable measure of error for analyzing the performance of diffusion maps
and PCA. Hence, we use the clean firing rate data set as our reference to determine how well both algorithms perform
on the noisy previous time data set.\\

First, we sample from the firing rate function  $R^{j}(t),$ for each time point $t \in [0, T]$,
to obtain vectors  $(R^{1}(t),  R^{2}(t),    \ldots , R^{N}(t) )  \in \R^N),$ that form each row of our  
clean synthetic firing rate data  matrix.\\


The second data set is generated as follows: we assume that the experiment consists of a single trial and that the spike train  $\{ t^j \}   =  \{t_{1}^{j}, ....., t_{n_{i}}^{j} \} $, of the $j^{th}$, neuron is generated according to a non-homogeneous poisson process with firing rate  $R^{j}(t),$ in the interval $[0, T].$ We then generate a spike  $t^j_i,$ with probability $\text{R}^{j}(t)\Delta t \ll 1$, otherwise no spike is generated.  Next, we define the previous time function (see section 4.3 )  based on the  generated set of synthetic spike trains.  We sample from the previous time function to obtain  vectors
$( \text{P}^{1}(t), \text{P}^{2}(t), \ldots, \text{P}^{N}(t) ) \in  \R^N,$ for each row of our previous time data matrix.\\

From Figure \ref{fig:PrevTime}, we  see that the previous time function is undefined at the beginning of the experiment where there is no recorded spike. In our analysis, we insert a spike before the start of the experiment by computing the mean previous time obtained after running  one simulation of  the previous time data. This ensures that the previous time function
is defined for all time (see discussion in section 6 for future directions). \\

At  each time point, $t  \in [0,T]$, the position of the animal, $\theta(t)$, is associated with a previous time function, $\text{P}^{j}(t)$, and a firing rate function, $R^{j}(t)$. We analyze both data sets using the diffusion maps algorithm  and  PCA.


\subsection{Distance  and similarity measure used in results section}
Given two time points $t_{1}$ and $t_{2}$ , we write the corresponding previous time vectors 

$$ \vect{P}(t_1)= (\text{P}^{1}(t_{1}), \text{P}^{2}(t_{1}), \ldots, \text{P}^{N}(t_{1}))   \  \ \text{and} \  \ 
 \vect{P}(t_{2}) = (\text{P}^{1}(t_{2}), \text{P}^{2}(t_{2}), \ldots, \text{P}^{N}(t_{2}))  \  \ \text{in} \  \  \R^{N}.$$

We use the l$_{1}$ norm to compute the distance between the two vectors:

\begin{equation}\label{distPrevtime}
\text{d}(\vect{P}(t_{1}), \vect{P}(t_{2}) ) = 
\displaystyle \sum_{j=1}^{N} \abs{ \text{P}^{j}(t_{1}) - \text{P}^{j}(t_{2})   }.
\end{equation}

Similarly, we use the $l_{1}$-norm to compute the distance between two firing rate vectors:

\begin{equation}\label{distFirerate}
\text{d}(\vect{R}(t_{1}), \vect{R}(t_{2}) ) = 
\displaystyle \sum_{j=1}^{N} \abs{ R^{j}(t_{1}) - R^{j}(t_{2})   }.
\end{equation}

We use the  l$_1$ norm  instead of the l$_{2}$ norm because the latter emphasizes large differences between data points which diverges from our goal of learning similar brain states  based on pairwise distances between spike trains.\\


We form pairwise similarities, $w_{ij}$, between each data point using the Gaussian kernel  which we eventually use as weights in the diffusion maps algorithm. The weights are given by,
\[
w_{ij} = e^{-\dfrac{\text{dist}^{2}(\vect{x}_{i}, \vect{x}_{j})}{2\sigma^2}} 
\]   where $\text{dist}(\vect{x}_{i}, \vect{x}_{j})$ is given by equations
\eqref{distPrevtime} and \eqref{distFirerate}. 
























%\begin{itemize}

%========suggested by Duane=========================================
%\item  First mention that there is a general conceptual framework
%namely dimensionality reduction and then say that
% the metric a approach is just one of the ways of doing dim reduction
% yet ensuring minimal information loss.

%\item The previous/next time approach is our new way of preprocessing
% the data and then using some of the usual metrics to analyze it.

%============end suggestion=========================================
%\item How did you create the metric and what precisely is it's definition?
%\item Experimental design. Discuss how the data was collected using a diagram
%\item Mention that this kind of analysis has never been 
%applied to this Redish Lab data.
%\item describe your methodology starting with the fake brain model and then tests on real data
%\item what are the names and definitions of the algorithms you're using?
%\item what is dimensionality reductions and why are you using linear instead of non linear
%\item Why did you choose this particular algorithm(s)?
%\item What is wrong with using PCA or Kernel PCA?
%
%\end{itemize}





