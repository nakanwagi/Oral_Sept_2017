%\mychapter{3}{methodology}

\section{Current Project}

\subsection{Our framework}

We propose a method of pre-processing spike times by looking at the time since the previous spike, which we refer to as the previous time. 
Preliminary simulations reveal that previous time vectors capture changes in spike timing during the four and a half laps taken by the animal around a simulated circular truck whereas using the traditional firing rate makes no such distinction. We use the diffusion maps algorithm \cite{coifman2006diffusion}, together with the $l_{1}$ distance, to obtain a low dimensional model of the simulated data.\\

%The measure of ``goodness" of our low dimensional model is done by comparing the mutual information  \cite{quiroga2009extracting, Dayan2001}  between the resultant eigen vectors and and the position of the animal. 


A mathematical model based on the point-process framework is an appropriate approach for modeling  data from the CA1 region of the rat hippocampus for two reasons. First, the position of the animal is coded by place cells using both the firing rate and the precise time at which the cells fire with respect to the hippocampal theta rhythm.
The theta rhythm is a sinusoid of frequency 7-12 Hz which occurs whenever a rat changes position in  a specific direction \cite{OKeefe1971, Burgess1993}.
Second,  metrics based on the point-process viewpoint are often non-Euclidean \cite{Aronov2004, Victor2005}. Moreover, there are specific examples like sensory space in  the olfactory system and the perceptual space of color vision that are typically non-Euclidean, making analysis in the point-process framework most appropriate.\\


\subsection{Nature of our raw real-world data}
The raw data is a set of multiple single-unit single-trial spike trains,\\
\[ 
\text{T}^{\text{spike}} = \displaystyle \{ \{ t_{i}^{j} \} , 1 \leq i \leq n_{i}, 1 \leq j \leq 32 \}  
\]
recorded from 32 neurons, called  place cells, in the CA1 region of the rat hippocampus, where,
$\displaystyle  \{t_{i}^{j}\} =  \{t_{1}^{j}, ....., t_{n_{i}}^{j} \} $, represents  a sequences of n$_{i}$ recorded times at which the spikes of the $j^{th}$ neuron occurred. See section 3.


\subsubsection{Preprocessing raw spike train data}
We pre-process the raw spike trains using two methods: In the first method, the raw spike trains are smoothed with a Gaussian kernel to obtain the firing rate (see section 3.4.2). In the second method, we compute the previous time defined in section 4.2.3.\\


\subsubsection{Conversation to a firing rate}
For each neuron labeled $j$,  we smooth the corresponding spike train T$^{\text{spike}}_{j}(t)$ to obtain the firing rate function $R^{j}(t).$ The  firing rate function, for the $j^{th}$ neuron is given by

\begin{equation} \label{jfirerate}
\text{R}^{j}(t) = \displaystyle \sum_{i=1}^{n_{i}}  \frac{1}{\sigma \sqrt{2\pi}} 
e^{-\dfrac{(t_{i_{k}}^{j}  - t_{i_{l}}^{j})^2}{2\sigma^2}}. 
\end{equation}

\subsubsection{The previous time function}
Given any time $t$, define the previous time of the $j^{th}$ neuron, denoted by $\text{P}^{j}(t)$ as follows. Let 
\[
t^{j}_{\text{prev}}(t) = \displaystyle \max  \{  t^{j}_{i} \ \ | \ \ t^{j}_{i} < t, 1 \leq i \leq n_{i} \} \ \ \text{for all spike trains} \quad  \{t^{j}_{i}\} \in T^{\text{spike}}.
\]
The previous time function, P$^{j}(t)$, is given by, 
\begin{equation}\label{prevtimefun}
\text{P}^{j}(t) = t^{j}_{\text{prev}}(t) - t.
\end{equation}


Figure \ref{fig:CellActivity_Prevtime} below illustrates how the previous time function is obtained from the firing activity of a given neuron.


 \begin{figure}[H]
        \centering
        \begin{subfigure}[b]{0.8\textwidth}
            \centering
            \includegraphics[width=\textwidth]{/home/tesylvia/Oral_Sept_2017/images/CellActivityGraph.pdf}
            \caption[Cell activity]%
            {{\small Cell activity}}    
            \label{fig:Cell activity}
        \end{subfigure}
        
        \vskip\baselineskip
        \begin{subfigure}[b]{0.8\textwidth}   
            \centering 
            \includegraphics[width=\textwidth]{/home/tesylvia/Oral_Sept_2017/images/PrevTime_graph.pdf}
            \caption[]%
            {{\small Previous time function}}    
            \label{fig:Prevtime}
        \end{subfigure}
        \caption[The previous time function for a single neuron ]
        {\small The previous time function based on the activity of a single neuron} 
        \label{fig:CellActivity_Prevtime}
  \end{figure}












\subsubsection{Specific implementation}
We consider a population of $N=32$ sensory neurons (place cells) encoding a 
one-dimensional circular stimulus variable, $\theta(t)$, which represents 
the position of the rat, at any time $t$ along a circular track, during
a behavioral task.
We let $\theta(t) = c t \ \ (\text{mod} \ \ 2\pi)$, where time $t$, is in seconds, $c = \dfrac{2\pi}{T_{\text{lap}}}$ is the speed of the animal, and
$\text{T}_{\text{lap}}$ is the time taken for the animal to make one lap around the circle.
For this synthetic experiment, we assume that information about the animal's
environment is encoded by the firing rate, $\text{R}^{j}(t).$
We model the place field of the $j^{th}$ neuron using a Gaussian 

\begin{equation}
{g}^{j}(\theta) = \displaystyle  f_{\text{bg}} + f_{\max} 
\exp\bigg(-\dfrac{\text{dist}^{2}(\theta - \phi_{j})}{2\sigma_{j}^{2}} \bigg)
\end{equation}



where $f_{bg}$ is the background firing rate that is independent of the underlying stimulus, $f_{\max}$ denotes the maximum firing rate of each 
neuron and $\phi_{j}$ is the preferred position of the $j^{th}$ 
neuron or the center of the j$^{th}$ place field.
Given any two angles, $\theta_{1}, \theta_{2} \in [0, 2\pi]$, the quantity
$\text{dist}(\theta_{1}, \theta_{2})$ denotes the shortest distance between 
two points on a unit circle, given by
\[
\text{dist}(\theta_{1}, \theta_{2}) = \abs{\big( \big( [\theta_{1} - \theta_{2}] + \pi  \big) \ \ \text{mod} \ \ 2\pi \big) - \pi}.
\]
The experiment consists of a single trial, in which the spike train of the $j^{th}$, neuron is generated according to a non-homogeneous poisson process with firing rate $R^{j}(t)$ defined by $R^{j}(t) = g^{j}(\theta(t))$ in the interval $[0, T]$. Choosing $m$ time points and setting the step size $\Delta t = \frac{\text{T}}{m}$, we divide the interval $[0, T]$ into short sub
intervals $[t_{i}, t_{i} + \Delta t]$ where $t_{i} = i \Delta t$, and generate
a spike with probability $\text{R}^{j}(t)\Delta t \ll 1$, otherwise no spike is generated.
For each time $t_{i}$, we sample from  the firing rate function \eqref{jfirerate} and  previous time function \eqref{prevtimefun} respectively, to obtain the firing rate vectors 
\[ \vect{r}(t_{i}) = (r^{1}(t_{i}), r^{2}(t_{i}), \ldots, r^{N}(t_{i}))^{\top},
\]
and previous time vectors
\[
\vect{p}(t_{i}) = (p^{1}(t_{i}), p^{2}(t_{i}), \ldots, p^{N}(t_{i}))^{\top}.
\]
Thus at each time, $t_{i} \in [0,T]$, the position of the animal, $\theta(t_{i})$, is associated with a previous time vector, $\vect{p}(t_{i})$,
and a firing rate vector, $\vect{r}(t_{i})$.
These vectors are our data points.  Here, N=32 since we are analyzing simulated data from 32 multiple single-unit single-trial recordings.



\subsection{Distance measure used in results section}
Given two time points $t_{1}$ and $t_{2}$ representing different fake brain states, we form the corresponding previous time vectors $\vect{p}(t_{1})= (p^{1}(t_{1}), p^{2}(t_{1}), \ldots, p^{32}(t_{1}))^{\top} $  and  $\vect{p}(t_{2}) = (p^{1}(t_{2}), p^{2}(t_{2}), \ldots, p^{32}(t_{2}))^{\top}$ in $\R^{32}$.

We use the l$_{1}$ norm to compute the distance between the two vectors:

\begin{equation}\label{distPrevtime}
\text{d}(\vect{p}(t_{1}), \vect{p}(t_{2}) ) = 
\displaystyle \sum_{i=1}^{32} \abs{ p^{i}(t_{1}) - p^{i}(t_{2})   }.
\end{equation}

Similarly, we use the $l_{1}$-norm to compute the distance between two firing rate vectors:

\begin{equation}\label{distFirerate}
\text{d}(\vect{r}(t_{1}), \vect{r}(t_{2}) ) = 
\displaystyle \sum_{i=1}^{32} \abs{ r^{i}(t_{1}) - r^{i}(t_{2})   }.
\end{equation}


\subsubsection{Similarity measure used in results section}
We form pairwise similarities, $s_{ij}$, between each data point
using the Gaussian kernel 
\[
s_{ij} = e^{-\dfrac{\text{dist}^{2}(\vect{x}_{i}, \vect{x}_{j})}{2\sigma^2}} 
\]
where $\text{dist}(\vect{x}_{i}, \vect{x}_{j})$ is given by equations
\eqref{distPrevtime} and \eqref{distFirerate}.
We then apply diffusion maps to the resultant similarity matrix S = (s$_{ij}$) 
as described in section 3.3.8, to obtain a low dimensional model for the 
firing rate and previous time data.
The two parameters in diffusion maps are $\sigma$ and $\alpha$.
We set $\sigma =1$ and $\alpha = 0.5$ for all our synthetic results.























%\begin{itemize}

%========suggested by Duane=========================================
%\item  First mention that there is a general conceptual framework
%namely dimensionality reduction and then say that
% the metric a approach is just one of the ways of doing dim reduction
% yet ensuring minimal information loss.

%\item The previous/next time approach is our new way of preprocessing
% the data and then using some of the usual metrics to analyze it.

%============end suggestion=========================================
%\item How did you create the metric and what precisely is it's definition?
%\item Experimental design. Discuss how the data was collected using a diagram
%\item Mention that this kind of analysis has never been 
%applied to this Redish Lab data.
%\item describe your methodology starting with the fake brain model and then tests on real data
%\item what are the names and definitions of the algorithms you're using?
%\item what is dimensionality reductions and why are you using linear instead of non linear
%\item Why did you choose this particular algorithm(s)?
%\item What is wrong with using PCA or Kernel PCA?
%
%\end{itemize}





