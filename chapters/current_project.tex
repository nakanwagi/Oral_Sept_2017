%\mychapter{3}{methodology}

\section{Current Project}

\subsection{Our framework}
The goal of our current project  is to analyze real-world spike time data without assuming a particular relationship between spikes and external variables. For example, one target data set is  obtained by measuring single units (place cells)  from the CA1 region of the rat hippocampus during a behavioral experiment in which a rat is performing a spatial task along a circular track (see figure \ref{fig:RasterPlot}). As place cells fire predominantly when the rat is in  particular regions \cite{o1971hippocampus, Burgess1994}, neural activity is correlated with space. However, our goal is an analysis approach that does not assume such correlation but instead discovers structure in the neural data alone. Such an analysis would then be applicable to spike time data sets where such a strong relationship with external variables is not evident.


We hypothesize that the high dimensional spike time data from neural activity may lie along a low dimensional manifold 
as the rat is moving along a one-dimensional track.  Therefore, we use dimensionality  reduction to learn the underlying
structure of the data (see section 3.3). We then  check to see how the low dimensional structure learned from the 
data itself corresponds to external variables such as the animal's position along the track.


To do dimensionality reduction, we need a distance measure  (see  section 3.4) to model pairwise relations between spike trains since our data is spikes. The dimensionality reduction algorithms we use require a summary of pairwise distances between spike trains as an input in order to learn the low dimensional model.


We propose a novel method of preprocessing spike times by looking at the time since the previous spike, which we refer to
as the previous time measure (see section 4.3). The previous time function (see section 4.3) captures the history of neural activity even in long intervals between spikes.


Since we do not have a suitable measure of error to analyze the performance of Diffusion maps and PCA on our
real-world spike data, we design a synthetic experiment (simulation)  as a first step towards  developing and testing the effectiveness of our preprocessing technique on spike data (see section 4.4-4.5 ).
To determine the performance of Diffusion maps and PCA on synthetic data, we carry out  our analyses  using two different synthetic data sets:  a clean firing rate data set and a data set consisting of spikes generated as a Poisson process.
The generated spike data is then preprocessed using our previous time measure before carrying out dimensionality reduction. We use the clean firing rate data set as our reference to determine how well both algorithms perform
on the  synthetic data.


Our preliminary results on synthetic experiments show that spike time data preprocessed by the previous time measure
capture the simulated four and a half laps taken by a rat around a circular track. We use the Diffusion maps algorithm 
to obtain a low dimensional model of the synthetic data and compare the algorithm's performance with that of PCA.


The sections that follow as organized as follows: In section  4.2 and 4.3,  we define the previous time function and illustrate it graphically.  In addition, we give a description of our simulation and analyzes used. In section 5, we show our results of diffusion maps and PCA on synthetic data sets. Finally in section 6, we give a brief discussion on our results and suggest possible future directions.

\subsection{Nature of spike  data}
The  spike  data  set consists  of multiple single-unit single-trial spike trains,

\[ 
\text{T}^{\text{spike}} = \displaystyle \{ \{ t^{j} \} ,  1 \leq j \leq N \}  
\]
recorded from $N$  neurons  where,
$\displaystyle  \{t^{j}\} =  \{t_{1}^{j}, ....., t_{n_{i}}^{j} \} $, represents  a sequences of n$_{i}$ recorded times at which the spikes of the $j^{th}$ neuron occurred. 

%However, we only show results based on synthetic data due to lack of a suitable measure of error to quantify 
%the performance of  diffusion maps and PCA on real world data.  We pre-process the simulated spike trains by computing  the previous time  defined in section 4.3.
 

%%%%%%%%%%%%%%%%%%%%%%%%%%%%%%%%%%%%%%%%%%%%%%%%%%%%%%%%%%%%%%%%%%%%%%%
%%%%%%% %%%  deleted this because our firing rate does not have spikes   %%%%%%%%%%%%%%%%%%%%%%%%%%
%\subsubsection{Conversation to a firing rate}
%For each neuron labeled $j$,  we smooth the corresponding spike train T$^{\text{spike}}_{j}(t)$ to obtain the firing rate function $R^{j}(t).$ The  firing rate function, for the $j^{th}$ neuron is given by
%\begin{equation} \label{jfirerate}
%\text{R}^{j}(t) = \displaystyle \sum_{i=1}^{n_{i}}  \frac{1}{\sigma \sqrt{2\pi}} 
%e^{-\dfrac{(t_{i_{k}}^{j}  - t_{i_{l}}^{j})^2}{2\sigma^2}}. 
%\end{equation}
%%%%%%%%%%%%%%%%%%%%%%%%%%%%%%%%%%%%%%%%%%%%%%%%%%%%%%%%%%%%%%%%%%%%%%%%

\subsection{The previous time function}
Previous time refers to the time since the last spike of a given neuron. The previous time function captures the history of the neuron's  activity even in long intervals between spikes.
Given any time $t$, define the previous time function of the $j^{th}$ neuron, denoted by $\text{P}^{j}(t)$ as follows. Let 
\[
t^{j}_{\text{prev}}(t) = \displaystyle \max  \{  t^{j}_{i} \ \ | \ \ t^{j}_{i} < t, 1 \leq i \leq n_{i} \} \ \ \text{for all spike trains} \quad  \{t^{j}\} \in T^{\text{spike}}, 
\]
where $t^j_i$ denotes the $i^{th}$ spike of the $j^{th}$ neuron.  The previous time function, P$^{j}(t)$, is given by, 
\begin{equation}\label{prevtimefun}
\text{P}^{j}(t) = t^{j}_{\text{prev}}(t) - t.
\end{equation}


Figure \ref{fig:PrevTime} below illustrates how the previous time function is obtained from the firing activity of a given neuron.


 \begin{figure}[H]
        \centering
          \includegraphics[width=\textwidth]{./images/PrevTime_graph.pdf}
           \caption[]
            {\small  The previous time function $\text{P}^{j}(t) = t^{j}_{\text{prev}}(t) - t,$ based on a raster plot of a single neuron.   The ticks on the raster plot represent the spike time of a neuron.  The previous time function captures the history of neural activity even in long intervals between spikes.  $P^{j}(t)$ is equal to zero when a neuron fires a spike  and then moves away from zero with a slope of negative one to a height equal to the difference between two adjacent spike times, below the horizontal axis.  } 
             \label{fig:PrevTime}
  \end{figure}


\subsection{Simulation}
We consider a population of $N=32$  neurons (idealized place cells) encoding a 
one-dimensional circular  variable, $\theta(t)$, which represents 
the position of a rat, at any time $t$ along a circular track, during a behavioral task  (cf.  figure \ref{fig:RasterPlot}).
In the simulation, we let the rat move at a constant speed $c$, setting  $\theta(t) = c t \ \ (\text{mod} \ \ 2\pi)$, where time $t$, is in seconds. The quantity  $c = \dfrac{2\pi}{T_{\text{lap}}},$ is the speed of the animal, and
$\text{T}_{\text{lap}}$ is the time taken for the animal to make one lap around the circle.
From  $\theta(t)$,  we model the place field of the $j^{th}$ neuron using a Gaussian 

\begin{equation}
{g}^{j}(\theta) = \displaystyle  f_{\text{bg}} + f_{\max} 
\exp\bigg(-\dfrac{\text{dist}^{2}(\theta - \phi_{j})}{2\epsilon_{j}^{2}} \bigg)
\end{equation}

where $f_{bg}$ is the background firing rate that is independent of the underlying stimulus, $f_{\max}$ denotes the maximum firing rate of each neuron, $\epsilon_j $  is the width of the $j^{th}$ neuron's place field ($\epsilon_j$ is chosen uniformly at random in  $[0.01, 1]$)  and $\phi_{j}$ is the preferred position of the $j^{th}$  neuron or the center of the j$^{th}$ place field. The centers of the place fields are equally spaced, that is, we set
$$\phi_j = \frac{j2\pi}{N}, \ \    \text{for}   \ \    1 \leq j \leq N.$$
Intuitively, $g^{j}(\theta)$, models  the likelihood that the rat's neuron fires given the animal's position $\theta(t)$, in space.
The Gaussian function indicates that the rat's place cell is  likely to fire maximally  at the center of the cell's place field
which mimics a characteristic  of a real-world rat's place cells.
Our model assumes that each neuron in the simulated rat's brain has a single place field. This is a realistic assumption as 
most place cells have one place field (see section 6 on possible future directions). 


Given any two angles, $\theta_{1}, \theta_{2} \in [0, 2\pi]$, the quantity
$\text{dist}(\theta_{1}, \theta_{2})$ denotes the shortest distance between 
two points on a unit circle, given by
\[
\text{dist}(\theta_{1}, \theta_{2}) =  \left | \big( \big( \theta_{1} - \theta_{2} + \pi  \big) \ \ \text{mod} \ \ 2\pi \big) - \pi \right|.
\]

Our simulation assumes  that information about the animal's environment is  encoded by the firing rate function
$$R^{j}(t) =  \text{g}^{j}(\theta(t)).$$
Thus given a rat position $\theta(t)$, all the $N$ neurons  in the simulated rat's brain fire according to the firing rate functions  $R^{j}(t),$ for $j = 1, \ldots, N.$


\subsection{Analysis}
We set the start  and end time of the synthetic experiment  to be  $0$ and $T$ respectively. We then divide the interval
$[0, T],$ into equal sub-intervals  of width $\Delta t$.
We carry out  our analyses  using two different data sets:  a clean firing rate data set and a data set consisting of spikes generated as a Poisson process.
We use two data sets because we do not have a suitable measure of error for analyzing the performance of diffusion maps
and PCA. Hence, we use the clean firing rate data set as our reference to determine how well both algorithms perform
on the spike time data set.


First, we sample from the firing rate function  $R^{j}(t),$ for each time point $t \in [0, T]$,
to obtain one data point  $(R^{1}(t),  R^{2}(t),    \ldots , R^{N}(t) )  \in \R^N.$ 


To obtain the second data set, we generate spikes  for a single trial. The spike times $\{t_{1}^{j}, ....., t_{n_{i}}^{j} \},$  of the $j^{th}$ neuron are generated as a non-homogeneous Poisson process at a rate $R^{j}(t),$ in the interval $[0, T].$
Next, we define the previous time function (see section 4.3 )  based on the  generated set of  spike trains.  
For each time point $t \in [0, T],$ we sample from the previous time function $\text{P}^{j}(t),$ to obtain  one data point
$( \text{P}^{1}(t), \text{P}^{2}(t), \ldots, \text{P}^{N}(t) ) \in  \R^N.$ 


From Figure \ref{fig:PrevTime}, we  see that the previous time function is undefined at the beginning of the experiment where there is no recorded spike. In our analysis, we insert a spike before the start of the experiment by computing the mean previous time obtained after running  one simulation of  the previous time data. This ensures that the previous time function
is defined for all time (see discussion in section 6 for future directions). 


At  each time point, $t  \in [0,T]$, the position of the animal, $\theta(t)$, is associated with a previous time function, $\text{P}^{j}(t)$, and a firing rate function, $R^{j}(t)$. We analyze both data sets using the diffusion maps algorithm  and  PCA. Note that the value of $\theta(t)$ does not enter these analyzes.


\subsection{Distance  and similarity measure used in results section}
Given two time points $t_{1}$ and $t_{2}$ , we write the corresponding previous time vectors 

$$ \vect{P}(t_1)= (\text{P}^{1}(t_{1}), \text{P}^{2}(t_{1}), \ldots, \text{P}^{N}(t_{1}))   \  \ \text{and} \  \ 
 \vect{P}(t_{2}) = (\text{P}^{1}(t_{2}), \text{P}^{2}(t_{2}), \ldots, \text{P}^{N}(t_{2}))  \  \ \text{in} \  \  \R^{N}.$$

We use the l$_{1}$ norm to compute the distance between the two vectors:

\begin{equation}\label{distPrevtime}
\text{d}(\vect{P}(t_{1}), \vect{P}(t_{2}) ) = 
\displaystyle \sum_{j=1}^{N} \abs{ \text{P}^{j}(t_{1}) - \text{P}^{j}(t_{2})   }.
\end{equation}

Similarly, we use the $l_{1}$-norm to compute the distance between two firing rate vectors:

\begin{equation}\label{distFirerate}
\text{d}(\vect{R}(t_{1}), \vect{R}(t_{2}) ) = 
\displaystyle \sum_{j=1}^{N} \abs{ R^{j}(t_{1}) - R^{j}(t_{2})   }.
\end{equation}

We use the  l$_1$ norm  instead of the l$_{2}$ norm because the latter emphasizes large differences between data points which diverges from our goal of learning similar brain states  based on pairwise distances between spike trains.


We form pairwise similarities, $w_{ij}$, between each data point using the Gaussian kernel  which we eventually use as weights in the diffusion maps algorithm. The weights are given by,
\[
w_{ij} = e^{-\dfrac{\text{dist}^{2}(\vect{x}_{i}, \vect{x}_{j})}{2\sigma^2}} 
\]   where $\text{dist}(\vect{x}_{i}, \vect{x}_{j})$ is given by equations
\eqref{distPrevtime} and \eqref{distFirerate}. 
























%\begin{itemize}

%========suggested by Duane=========================================
%\item  First mention that there is a general conceptual framework
%namely dimensionality reduction and then say that
% the metric a approach is just one of the ways of doing dim reduction
% yet ensuring minimal information loss.

%\item The previous/next time approach is our new way of preprocessing
% the data and then using some of the usual metrics to analyze it.

%============end suggestion=========================================
%\item How did you create the metric and what precisely is it's definition?
%\item Experimental design. Discuss how the data was collected using a diagram
%\item Mention that this kind of analysis has never been 
%applied to this Redish Lab data.
%\item describe your methodology starting with the fake brain model and then tests on real data
%\item what are the names and definitions of the algorithms you're using?
%\item what is dimensionality reductions and why are you using linear instead of non linear
%\item Why did you choose this particular algorithm(s)?
%\item What is wrong with using PCA or Kernel PCA?
%
%\end{itemize}





