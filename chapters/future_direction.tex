%\mychapter{7}{Discussion}
\section{Future direction}
First, at the beginning of the experiment, there are no recorded spike times. However, there must be at least one spike before  the previous time function is defined. This raises a problem on how to address the boundary condition at the beginning of the experiment. As it stands, we have addressed this problem by first running the experiment once to obtain the mean of the sampled previous time data, and afterwords, we insert a spike before the beginning of the experiment based on the computed previous time average, and then use that to simulate new  previous time data. We think that this is not the best approach, as the mean may not be the best representation of the underlying distribution. We plan to find another way of addressing the boundary condition for the previous time vectors. For instance, we plan to insert a spike before the beginning of the experiment based on samples drawn from a correctly estimated underlying model of the data.\\

Second, we currently do not have a measure of error for analysing the  effectiveness of our approach. 
We tried using Shannon mutual information on real-world data. However, we found that  high  mutual information may be caused by spatial noise, such as changes in the animal's head direction rather than indicating that the analysis captured the animal's movement around the track.  We plan to use the Fisher information as a goodness of measure in future work since it can take advantage of the fact that results lie in a continuous space. In addition, Fisher information is highest when  the variance of the estimated model parameter is very small. This could  yield a more reliable estimate of
how well we captured a one-dimensional manifold that corresponds to the animal's position along the track. \\

We plan to preprocess  our real-world spike time  data  using the previous time measure  in order to look for
structure in data itself so as to see how the structure corresponds to external variables. The  data set consists of spike times
obtained by measuring from $N = 32$ single units (place cells)  from the CA1 region of a rat hippocampus during a behavioral experiment in which the rat is performing a spatial task along a circular track (see figure \ref{fig:RasterPlot}).\\

Finally, we plan to look at other types of distance measures for quantifying neural response variability such as variants of the kendal tau distance  or a weighted combination of some existing distances instead of the $l_1$ distance.\\

An improved approach such as the one we envision and outline here  may provide some insight on how to analyze
spike time data when the relationship between spikes and external variables is unknown.  By looking for structure in the 
spike time data itself,  a researcher may be able to see how the underlying structure corresponds to external variables such as a stimulus  or a location without imposing apriori ideas  on the relationship between these variables and the observed response.















%Our first step was to determine what dimensionality reduction algorithms to use.
%We decided to use Laplacian eigmaps \cite{belkin2003laplacian} and Diffusion Maps \cite{coifman2006} which are both non-linear dimensionality reduction algorithms.
%why?----to be addressed later.
%Initially, we smoothed the spike data using an exponential kernel defined by
%
%\[
%  K_{\tau}(t) =
%  \begin{cases}
%                 \frac{1}{\tau} \exp(-\frac{t}{\tau}) & \text{if $t > 0$} \\
%                  0 & \text{elsewhere} 
%  \end{cases}
%\]
%
%\subsection{Observations}
%We found that using Laplacian eigen maps to get an embedding based on the firing rate tried to
%recovered the position of the rat but could not reflect any variations along the path
%e.g the animal could have looked away from the track or run in a ragged fashion around the track.\\
%
%We also found that whenever there were gaps between the receptive fields (cases with no spikes),Laplacian Eigen Maps (LAM) performed poorly.
%This is because the nearest neighbor graph (based on the Euclidean distance used in (LAM) yields several connected components (i.e, the graph is disconnected).
%Since the eigen value decomposition step  in LAM is only applied on the largest connected
%component of the graph, the eigen vectors output by LAM are shorter than the total number of original data points (spike times). Thus the embedding provided by LAM is in accurate in some
%instances due to the "coverage" problem.\\
%
%Diffusion Maps (DM) algorithm tends to run out of memory in case of large instances
%so we were unable to compare the performance of both LAM and DM when using the firing rate.
%This is not true: First redo the analysis on sonet since you need to show that
%previous/next time is an improvement over the traditional firing rate.
%
%\subsection{Remedy for the coverage problem}
%Inspired by David Redish's idea, we decided to create previous and next time vectors which
%give us information even when there is no spike.
%This is the direction which seems promising at the moment since it tends to over come
%the problem of coverage. what is the coverage problem? (see reference on tunning curves).
%We then used the exponential kernel as our metric on the previous and next time data
%to generate a distance metric which is the main input in Diffusion Maps algorithm.
%
%
%
%
