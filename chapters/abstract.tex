% Abstract page

\thispagestyle{empty}

\begin{center}
    \Large
    
\section*{Abstract}


\end{center}

\vspace{1cm}


Our perception of the world is influenced by the way our brains processes information received from millions of neurons
in our nervous system. Even if what we hear, see or feel may vary depending on our environment, information sent to the brain as a result of neural activity consists of a sequence of identical electrical pulses often referred to as spikes.
In this paper, we attempt to answer a central question: how do we understand and analyze neural responses when the relationship between the way neurons encode information and external variables such as stimuli, location or behavior  is unclear?


Our main innovation is  that we  first ignore these external variables and instead look for structure solely within data representing neural activity such as spike trains. Our preliminary results on synthetic data show that diffusion maps using spike time data preprocessed by our novel previous time measure, captures the  one-dimensional manifold corresponding to a simulated rat's movement around a track. In otherwords, diffusion maps  reveals the structure of a 
one-dimensional  manifold expected from the fact that the neural activity is strongly correlated with the rat's position.




Notes before submission: Incorporate the corrections that Duane emailed you before submission.........

  \vfill





\newpage