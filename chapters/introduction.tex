
\section{Big picture}
 
Examining stimulus-response data and extrapolating from that a model
that allows us to predict, given a set of responses, what the stimulus 
likely was.



%---------This should be more general and give a big picture-------------
%----what do you want to accomplish?
%----what is neural activity in general?
%-----why is a low dim model important?
%-----is idea to understand how the brain is working?
%------How is the brain coding information?
%------how can we decode information based on brain activity?
%----what consequences would there be if we found a low dimensional structure?
%----what is the relevance of having a similarity measure?
%-----e.g want to know whether the brain is thinking about similar things 
%----at different brain states/instances
%----why?  --- what does this have to do with the structure of the space



\section{Biological background}
\subsection{Structure of a neuron}
A neuron is a specialized cell in the nervous system that receives, represents, and transmits information through a series of electrical pulses called action potentials or spikes. The action potentials are propagated at uniform strength and speed but with varying frequencies. The neuron is the fundamental unit of brain function and is made of three major parts: the dendrites (which receive information from stimuli such as neurotransmitters), the cell body or soma (which processes information) and the axon (which transmits information to other neurons and other organs such as the brain) (Figure \ref{fig:Neuron}).

\begin{figure}[h]
\centering
\includegraphics[width=\textwidth]{/home/tesylvia/Oral_Sept_2017/images/Neuron.jpg}
%label the figure so latex can reference it
\caption{Structure of  a neuron}
      \label{fig:Neuron}
\end{figure}


The cell membrane is made up of phospholipids (fat) and separates the cell interior from the extracellular space. Embedded in the cell membrane are Na$^{+}$ (sodium) and K$^{+}$ (potassium) ion channels which pump out three Na$^{+}$ ions for every two K$^{+}$ ions pumped in. There are other ion channels (trans-membrane proteins) embedded in the cell membrane which
open or close (are gated)  enabling predominantly K$^{+}$, Na$^{+}$, Ca$^{2+}$ (calcium), and Cl$^{-}$ (chloride) ions to flow into and out of the cell. As a result, Na$^{+}$ is more concentrated outside the cell than inside it, and the intracellular concentration of K$^{+}$ is substantially higher than that outside the cell.
The lipid cell membrane is impermeable to charged ions but thin enough to allow interaction
of separated charged ions through electrostatic forces. 
Thus the cell membrane acts as an electrical capacitor whereas the gated ion channels act as conductors.

\subsection{Membrane potential}
A potential is a distribution of charge across the cell membrane.
\textit{Voltage} is a measure of the potential energy generated by separated charges and is measured in millivolts (mV). Ions flow into and out of the cell due to both voltage and concentration gradients. \textit{Current} refers to the flow of charged ions into and out of the cell. A resting neuron contains a greater number of negative charges on the inside than on the outside. 
This difference in separated charges is called the neuron's \textit{membrane potential}. A neuron with a membrane potential of approximately -70mV is called \textit{polarized}. This number is also referred to as the resting membrane potential. In a resting neuron, all the voltage-gated ion channels are closed. 

\subsection{Generation of an action potential}
Dendrites contain chemically-gated Na$^{+}$ channels, which open when 
a stimulus affects a sensory receptor, such as neurotransmitters binding to the dendrite receptors. This leads to an influx of Na$^{+}$ in the intracellular fluid, causing the membrane potential to be less negative or to be positive (depolarization of the neuron). 
This increase in the membrane potential causes the voltage-gated Na$^{+}$ channels, at the entry point of the axon, to open and thus more Na$^{+}$ flows into the cell down its electrochemical  gradient. When a certain threshold is reached ($\approx$ -55mV), an electrical pulse lasting  a short duration ($\approx$ 1ms),  called an \textit{action potential}, is released and is propagated over long distances along the neuron's axon. When the membrane potential rises (to  $\approx$ 40mV), the voltage-gated K$^{+}$ channels open, allowing more K$^{+}$ to flow out of the cell, which causes
the membrane potential to fall below the resting potential (hyperpolarization of the neuron). The neuron later returns to its resting potential after a refractory period, during which the likelihood of spiking is greatly reduced.
The axon terminal contains voltage-gated Ca$^{2+}$ channels, which open and release neurotransmitters into the synaptic cleft. The neurotransmitters bind to the dendrite receptors of nearby neurons.
A synapse is a specialized structure that facilitates communication between neurons. The neuron that sends off an action potential is called \textit{presynaptic} and the one receiving the 
chemical message is called a \textit{postsynaptic}.
Depending on the chemical properties of the binding neurotransmitters, action potentials fall into two broad categories. Excitatory post synaptic potentials (EPSP) result from excitation of a postsynaptic neuron, while inhibitory post synaptic potentials (IPSP) result from inhibition.

\subsection{Measuring neurons}
The electrical properties of biological cells are measured using electrodes, which allow electrical current to pass through them when they come into contact with electrolytes. Due to the small size of cells, microelectrodes are typically used to measure single unit spiking activity or stimulation of neurons.
A \textit{single unit} refers to a single action potential-generating neuron, whose spikes are clearly isolated by a recording microelectrode. (cite the reference for this and next figure).
The \textit{local field potential} (LFP) refers to the sum of numerous extracellular potentials generated by flow of ionic currents across the cell membrane, distributed along multiple neurons, due to the discharge of an action potential from a presynaptic neuron. LFPs represent summed average synaptic potentials in a given volume of the brain and can be used to characterize synaptic inputs as either excitatory or inhibitory.
Membrane potentials are measured using \textit{intracellular recordings}, a process where an electrode (such as a hollow glass electrode filled with conducting electrolyte) is inserted inside the cell body. The value of the membrane potential is obtained by comparing the potential on the inserted electrode to that of a reference electrode placed in the extracellular fluid surrounding the cell body.  The process in which an electrode is inserted in the extracellular space near the cell body is called \textit{extracellular recording} (Figure \ref{fig:Electrodes}).

\begin{figure}[h]
\centering
\includegraphics[width=\textwidth]{/home/tesylvia/Oral_Sept_2017/images/Electrodes.png}
\caption{Intracellular and  extracellular recordings, adapted from figure 2 of \cite{Humphrey1990}.}
%label the figure so latex can reference it
      \label{fig:Electrodes}
\end{figure}


Both extracellular and intracellular recordings are used to record potentials due to synaptic transmissions such as LFPs, and potentials due to firing of an action potential. Using electrolyte dyes, intracellular recording enables visualization
of cell structure and can differentiate between cell types.
Extracellular recordings are usually carried out in vivo while intracellular recordings can be carried out both in vivo and also in vitro preparations, such as slices of brain tissue. Multi-electrode arrays such as the $10 \times 10$ utah array enable simultaneous extracellular recordings from  multiple neurons in multiple brain sites. Intracellular recording, as in behaving animals such as rats, may be carried out using specialized cells called tetrodes.\\
Multi-unit single-trial recordings refer to simultaneous  extracellular recordings of neural activity from hundreds of cells.
For spike train analyses, the multi-unit single-trial recordings are usually processed using  spike sorting algorithms. Following spike sorting, neuronal types are identified by classifying the isolated units into known cell groups of the cortex.\\

The data set we are analyze is based on intracellular multiple single-unit single-trial recordings where the action potentials are measured from different single units using tetrodes. Tetrodes are used in the rat hippocampus where there is a dense network of cells, which would otherwise be hard to isolate using the usual spike sorting methods as in multi-unit single-trial recordings.


\subsection{Place cells and place fields}
Place cells are neurons found in the CA1 and CA3 region of the rat hippocampus,
whose firing rate increases maximally when the animal is in a specific location
in its environment, called a place field. A place cell tends to fire less often when the animal is outside its place field. Firing activities of place cells do not depend solely on specific stimuli but are rather a reflection of where the animal is in its environment \cite{OKeefe1971}. Place cells can also fire when an animal is in an area associated with a particular hidden task even if this area
is outside the animal's place field.


\subsubsection{Characteristics of place fields}
The characteristics of place fields \cite{OKeefe1978,Burgess1994} include the following:\\
First, a single portion of the animal's environment is represented by a group
of hippocampal place cells, a large portion of which is densely distributed
in the CA1 region of the rat hippocampus.
Second, place fields are not evenly distributed in the animal's environment.
Moreover, the distribution of place fields in space may depend on the type
of enclosure in which a behavioral task is carried out.
Third, there are no topographic relationships between place fields of two environments. Specifically, it is possible for distant place cells to have adjacent place fields and neighboring place cells do not necessarily have adjacent place fields.
Fourth, each place field is restricted to a portion of the animal's environment.
Within a structured environment like a maze in a laboratory, the place fields
tend to have a single-peaked shape and may be double-peaked on rare  occasions.
Fifth, since place fields can be determined by sensory cues from different
stimuli such as light and sound, place fields may rotate in space when the cues are rotated in a restricted environment.
Finally, place cell firing is closely correlated with the animal's (rat) position
in the environment \cite{1993Doth}.




















































%=========================================================================================


%=======Below was my original objective===================================================
%The objective of this present project is to find a low dimensional model of interactions, among a subtype of neurons called place cells, in the CA1 region of the rat hippocampus, believed to be specified to relay the animal's physical position. The model is extrapolated from application of a non-linear dimensionality tool called diffusion maps, to a designed 
%similarity matrix of activity patterns.\\



%======================Gaussian Processes below for future works==============================
%\newpage
%\section{Linear dimensionality reduction techniques}
%\subsection{Gaussian Process Factor analysis}
%\begin{Def}
%A vector-valued random variable $\vect{X} = \left[x_{1}, \ldots , x_{n} \right]^T$
%has a multivariate Gaussian distribution if it's  probability density function is given by
%   
%\[
%f(\vect{x})  = (2\pi)^{-\frac{n}{2}} \det({\Sigma})^{-\frac{1}{2}} 
%\exp \bigg( -\frac{1}{2}(\vect{x - \mu})^{T}\Sigma^{-1}(\vect{x - \mu}) \bigg)
%\]
%
%with mean vector  $\vect{\mu} \in \mathbb{R}^n$ and covariance matrix $\Sigma$.
%The covariance matrix must be a positive semidefinite (PSD) matrix for such a density to exist.
%We write $X  \sim N(\vect{\mu}, \Sigma).$
%
%\end{Def}
%
%
%\begin{Def} A Gaussian Process (GP) is a Gaussian distribution over functions \\
%$f: \mathbb{R}^n \rightarrow   \mathbb{R}^n$  defined by specifying a mean
%function $m: \mathbb{R}^n \rightarrow \mathbb{R}$  and a kernel
%$K: \mathbb{R}^n  \times \mathbb{R}^n \rightarrow \mathbb{R} $ such that the following
%conditions hold:
%
%\begin{itemize}
%\item each vector valued random variable 
%$f(\vect{t}) = \left[f(t_1), \ldots , f(t_n) \right]^T$ has a multivariate Gaussian distribution for all  
%$\vect{t} = \left[t_1, \ldots, t_n   \right]^T$, that is, 
%$f(\vect{t}) \sim  N(m(\vect{t}), K(\vect{t}, \vect{t})).$
%
%\item $m(\vect{t}) = \E(f(\vect{t}))$
%
%\item $K(\vect{t},\vect{s}) = \E\left[ \big(f(\vect{t}) - m(\vect{t}) \big) \big( (f(\vect{s}) - m(\vect{s}) \big)^T   \right]$  for any $\vect{t}, \vect{s} \in \R^n.$  
%
%\item K satisfies the Mercer theorem.
%
%\end{itemize}
%\end{Def}
%
%\begin{Thm}
%
%Every matrix $ K(\vect{t}, \vect{t}) = \{ K(t_i, t_j) \}_{1 \leq i, j \leq n} = (k_{ij}) $
%is a PSD  for all time $\vect{t}$ if
%\[  \vect{v}^T K \vect{v} = 
%\sum_{i=1}^{n} \sum_{j=1}^{n} k_{ij} v_i v_j =
% \sum_{i=1}^{n} \sum_{j=1}^{n} k(t_i, t_j) v_i v_j \geq 0  
% \quad \text{for all}  \quad \vect{v}  \neq \vect{0}. \] 
%
%\end{Thm}
%
%
%\begin{Ex}
%One of the most commonly used kernels is the squared exponential kernel (SE) given by
%$K(t_{i}, t_{j}) = \sigma_{f}^{2} \exp \{-\frac{1}{2l^2}  (t_i - t_j)^2 \}$
%where $\sigma_{f}^{2}$  is the variance of the kernel and l is the length scale.
%
%\end{Ex}
%



%============================================================================================
%%======How to insert a table in latex===================%

%\begin{table}[H]
%   \centering
%    \begin{tabular}{|c|c|c|c|}\hline
%    $x$ & 0 & 1 & 2\\ \hline
%    $f(x)$ & 3 & 6 & 9\\ \hline
%    \end{tabular}
%     \caption{Caption goes here}
%\end{table}

%%=====end table==================================%
%\begin{figure}[H]
%  \centering
%    \includegraphics[scale=0.5]{Bursting_neuron.png}
%     \caption{The bursting neuron}
%      \label{tab:data1}
%\end{figure}


%===========Insert a figure in latex=============%






%============end figure===========================%

%\begin{itemize}
%\item What is the nature of the problem you're trying to solve?
%\item Two most well known models that inspired your work
%\item The authors and names of these models
%\item What were they modeling?
%\item Strengths and weaknesses of these models
%\item Overview of our model
%\item What is promising about our model
%\item why are you going to use dimensionality reduction to study the model
%\item why use vector space embeddings instead of the point process framework
%\item Report any results that may be significant and supported by a measure of "goodness"
%\item Mention that this is the first time this type
%of analysis has been applied to the Redish Lab data 
%obtained from the CA1 region of the Hippocampus
%\end{itemize}

