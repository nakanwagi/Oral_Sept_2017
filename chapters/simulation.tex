\mychapter{5}{Simulation Results}

\section{Generation of Spikes}
In this section, we describe the standard procedure of estimating the 
firing rate and average spike count. We then use these estimates
to simulate spikes using a non-homogeneous poisson process.
Finally, we give a detailed description of the point process framework and 
state its advantages over using the estimated firing rate

\subsection{Derivation of the time varying firing rate}
Recall that for j=1, the neural response function $\eqref{spiketrain}$ represents a sequence of recorded times in an interval [0, T] at which a neuron fires an action potential (spike train)

\begin{equation}\label{response function}
 \rho(t) = \sum_{i=1}^{n} \delta(t-t_{i})  
\end{equation}
where n is the total number of spikes in the spike train and $t_{i}$ is the time at which the i$^{th}$ spike occurred. Using \eqref{DiracDelta}
We can write the total the number of spikes in the train by integrating the response function
\begin{align*}
\int_{\mathbb{R}}  \rho(t)  dt &=   \int_{0}^{T}  \rho(t)  dt\\
      &= \displaystyle  \sum_{i=1}^{n}    \int_{0}^{T}  \delta(t-t_{i}) dt\\
              & = \displaystyle  \sum_{i=1}^{n} 1 = \text{n}
\end{align*}


Thus the spike-count rate, r,  on the interval [0, T]  given by

\[ \dfrac{\text{the number of spikes over the time interval of length T }}{T} \]

can also be expressed as the average of the neural response function
over a single trial of duration T

\begin{equation}\label{spike-count rate}
  \text{r} =  \frac{1}{T}  \int_{0}^{T}  \rho(t)  dt
\end{equation}

We know that to reduce loss of temporal information due to averaging over a long period of time, the duration T of the experiment is often divided into small intervals [t, t+$\Delta$t] of length $\Delta$t so that the firing rate at time t,
denoted fr(t)  is  given by

\begin{equation}\label{single-trial emprical-spike count}
fr(t) = \dfrac{\text{number of spikes in the time interval} \quad [t, t+\Delta t]}
{\Delta t}
\end{equation}


\begin{Def}\label{trial-average neural response}
Let \text{M} be the total number of trials during an experiment
of multiple presentation of the same stimulus starting at time t=0 and to t=T. 
Let \[ \rho_{j}(t) = \displaystyle \sum_{i=1}^{n} \delta(t-t_{i})\] 
be the neural response function for the the j$^{th}$ trial for all j=1, $\dots$ M. The quantity 
\[ \langle \rho(t) \rangle  = 
\displaystyle  \frac{1}{M} \sum_{j=1}^{M} \rho_{j}(t)\] 

is called the trial-averaged neural response function

\end{Def}

The instantaneous time-varying firing rate, r(t), is the expectation of the 
trial-averaged neural response function over an infinite number of trials i.e.

\begin{equation} \label{expected FireRate}
r(t) = \displaystyle  \lim_{\Delta t \rightarrow 0}    \frac{1}{\Delta t}  \int_{t}^{t+\Delta t} 
 \langle \rho(t) \rangle \quad dt 
\end{equation}


In empirical experiments, the limit ${\Delta t \rightarrow 0}$ is usually
omitted since it's not possible to carry out an infinite number of experiments
in practice. Instead a sufficiently large number of trials is used
to estimate r(t), that is,

\[ 
r(t) \approx  \displaystyle  \frac{1}{\Delta t}  \int_{t}^{t+\Delta t} 
 \langle \rho(t) \rangle \quad dt 
\]

provided that there are enough spikes in the interval defining the r(t)
to yield a reliable estimate of the firing rate.

For a large $\Delta$t (long intervals), the average spike count denoted $\langle n  \rangle$ can then be  defined from the instantaneous firing rate as follows:

\begin{equation}\label{average spike count}
\langle n \rangle =  \displaystyle  \frac{1}{\Delta t}  
\int_{t}^{t+\Delta t}  r(t)  \quad dt 
\end{equation}

From equation \eqref{average spike count}, the average spike count is equivalent
to counting the number $n_{i}$  of spikes in the i$^{th}$ trial, for an infinitely large number of trials and taking the average of those spikes across the trials.


However for short intervals around t such as $(t-\Delta t, t+\Delta t)$, the average spike count    $\langle n \rangle$ is approximately equal 
to r(t)$\Delta$t for small $\Delta$t.
In addition, $\Delta$t can be made arbitrarily small such that
the probability of getting more than one spike in the interval
 $(t-\Delta t, t+\Delta t)$ is zero. 
Hence in this interval, the average spike count is precisely the probability of getting one spike for sufficiently small $\Delta$t.


\subsection{Probabilistic interpretation of firing rate}
The probability of getting one spike in a short time interval  $(t-\Delta t, t+\Delta t)$ is equal to the value of the instantaneous firing rate, r(t) during that interval multiplied by the length of that interval, that is,

\[ 
\displaystyle  \text{p}(\{ \text{one spike during the interval interval} \quad  (t-\Delta t, t+\Delta t)  \}) = \text{r}(t)\Delta t. 
\]

\subsection{Short comings of using the firing rate}
First, the firing rate is an average of the neural response function over
multiple trials which gives only a partial description of the neuron's responses
unlike the spike train itself.
Second, the method of averaging across multiple trials is contrary to the way
the sensory system operates in general.
For instance a sequence of action potentials is sent from the optic nerves
to the brain which sends impulses to motor neurons so that muscles in eyes can close an animal's eye lids when a fly gets close to the eye's pupil.
The nervous system does not wait for the fly to get close to the pupil multiple
times before our eye lids shut to such a threat.
Thirdly, the firing rate does not take into account other patterns of spikes
that convey information such as the inter-spike intervals.
Thirdly, there are several underlying neural mechanisms such as cognition, memory
and olfaction which the experimentalist cannot control and hence cannot be extensively studied by presentation of controlled stimuli.
Finally, the firing rate r(t) determines the probability of getting one spike
in a short time interval $(t-\Delta t, t+\Delta t)$ but is generally not sufficient to determine the probability of getting an entire sequence of n spikes
in that short time interval.




\subsection{The point process framework}
In the point-process framework, we view the neural response function as a point process. This implies that each spike train (a discrete event) consists of a sequence of spike times $\{t_{1}, \ldots, t_{n_{i}} \}$ which
are continuous random variables. Thus we know that each out come (spike train)
obtained during an intracellular recording is characterized by some underlying continuous probability density function. Denote this probability density function by p($\vect{t}$) where $\vect{t}$ denotes the vector of spike times $(t_{1}, \ldots t_{n})$. By the properties of a continuous random variable, the probability of obtaining a spike at time $t_{i}$ is equal to zero. Thus to obtain a non-zero probability, we seek the probability of obtaining a spike in a given time interval $(t_{i}, t_{i}+\Delta t)$ of length $\Delta$t.
Since we are assuming that each spike time $t_{i}$ is a stochastic random variable, it follows that  probability of getting one spike in a short time interval $(t_i, t_i+\Delta t)$ is equal to the product of  the probability density of $t_{i}$ also denoted by p(t$_{i}$) multiplied by the length 
of that interval ($\Delta$t), that is,

\[
\displaystyle \text{p}(\{\text{one spike during the interval} \quad (t_{i}, t_{i}+\Delta t) \}) = \text{p}(t_{i})\Delta t  
\]

\newpage

Thus the expected spike count rate, the probabilistic (theoretical) counter part
of \eqref{single-trial emprical-spike count}, is given by

\begin{equation}\label{expected count rate}
\displaystyle 
\dfrac{\text{p}(\{\text{one spike during the interval} \quad (t_{i}, t_{i}+\Delta t)}{\Delta t} 
\end{equation}

Taking the limit as $\Delta t \rightarrow 0$, we obtain the expected instantaneous  firing rate
 
\begin{equation}\label{expected instant Firerate}
\displaystyle 
\text{FR}(t) =  \lim_{\Delta t \rightarrow t}  \dfrac{\text{p}(\{\text{one spike during the interval} \quad (t_{i}, t_{i}+\Delta t)}{\Delta t} 
\end{equation}


The beauty of the point process framework is that it provides a way of determining the probability of obtaining a sequence of n spikes i.e spike train.

Indeed, viewing the neural response function as a point process, the
probability of getting a sequence of n spikes in a short time interval
$(t_{i}, t_{i}+\Delta t)$ for all $i=1, \ldots n$ is given by the
product of the joint probability density of the random n-dimensional
vector of spike times p($t_{1}, \ldots, t_{n}$) multiplied by the quantity
$(\Delta t)^{n}$\\


Still working on this to get to a non-homogeneous poisson process simulation.....




